\section{Documentation Plan}

\subsection{Style Guide}

All functions should clearly describe what task they perform and, as often as possible, should be between 5-6 lines long to minimize confusion. Variable names should also be unambiguous as to their contents and should refrain from using more than two dimensions for vector arrays. All functions should have a documentation string which provides a description of the function and, if necessary, descriptions of the incoming and outgoing variables. If possible, the variable names should make this clear without documentation.

\subsection{Programmer's Manual}

The programmers manual will detail what is required for the development of new material model user subroutines and will be 1-2 pages in length. Further documentation will be handled through the style of clear variable and function names as well as the function documentation strings.

\subsection{User's Manual}

The users manual will detail how to define a new finite element mesh, material properties, and how to run the simulation. This manual is unlikely to be longer than 1 page.

\subsection{Theory Manual}

The theory manual will provide a complete reference to the equations being used and their implementation in the code. The theory manual is provided in the form of a \LaTeX~presentation which will provide detailed derivation information.

It is currently intended for the test information to be included in the report rather than in the theory manual. This seems to be a more natural place for the inclusion of such information.

