%\documentclass[11pt]{article}
\documentclass{asme2ej}
\usepackage{amsmath,amssymb,graphicx,bm}
\usepackage{listings, color, subcaption, placeins}
\usepackage{undertilde}
\usepackage{algorithm,algpseudocode}
\usepackage{multicol}
\usepackage{makecell}
\usepackage[table]{colortbl}
\graphicspath{{./images}}
%%%%%%%%%%%%%%%%%%%%%%%%%%%%%%
%%%%%%%%%%%%%%%%%%%%%%%%%%%%%%
%%%%%%%%%%%%%%%%%%%%%%%%%%%%%%

%% If you want to define a new command, you can do it like this:
\newcommand{\Q}{\mathbb{Q}}
\newcommand{\R}{\mathbb{R}}
\newcommand{\Z}{\mathbb{Z}}
\newcommand{\C}{\mathbb{C}}
\newcommand{\e}{\bm{e}}
\newcommand{\TEN}[1]{\underline{\underline{#1}}}
\newcommand{\VEC}[1]{\utilde{#1}}
\newcommand{\UVEC}[1]{\underline{#1}}
\newcommand{\PK}[1]{\TEN{\tau}^{(#1)}}
\newcommand{\cauchy}{\TEN{\sigma}}
\newcommand{\st}{$^{\text{st}}$}
\newcommand{\nd}{$^{\text{nd}}$}
\newcommand\defeq{\mathrel{\stackrel{\makebox[0pt]{\mbox{\normalfont\tiny def}}}{=}}}

\graphicspath{{./images/}}

%% If you want to use a function like ''sin'' or ''cos'', you can do it like this
%% (we probably won't have much use for this)
% \DeclareMathOperator{\sin}{sin}   %% just an example (it's already defined)


\begin{document}
\title{Programmers Manual}
\author{Nathan Miller}

\maketitle

\begin{abstract}

This is intended to provide information on the major modules and functions contained within the user element and testing framework. Focus will be placed on requirements for the development of additional constitutive models and the connections of the UEL with Abaqus.

\end{abstract}

\tableofcontents

\section{fea\_driver.py}

\subsection{Description}

The overall driver code for the UEL. This code is considered to be largely auxillary to the fundamental aim of the project which is the development of a user element for Abaqus. It is useful however to verify that the element is working correctly outside of the confines of Abaqus itself. This allows for the element to be verified individually and then as a part of the whole.

\subsection{Classes}

\subsection{Functions}

\section{micro\_element.py}

\subsection{Description}

\subsection{Functions}

\section{micromorphic\_linear\_elasticity.py}

\subsection{Description}

\subsection{Functions}

\section{hex8.py}

\subsection{Description}

\subsection{Functions}

\FloatBarrier

\bibliographystyle{asme2ej}
\bibliography{micromorphic}

\end{document}